\documentclass[utf8]{article}
\usepackage[utf8]{inputenc}
\usepackage[T2A]{fontenc}

\usepackage[left=2cm,right=2cm,top=2cm,bottom=2cm,bindingoffset=0cm]{geometry}
    

\usepackage{natbib}
\usepackage{graphicx}

\begin{document}
\date{}
\begin{center}
\hfill \break
\large{Министерство образования и науки Российской Федерации}\\

\large{Государственное образовательное учреждение высшего профессионального образования}\\
\hfill \break

\large{\textbf{«ВОЛГОГРАДСКИЙ ГОСУДАРСТВЕННЫЙ ТЕХНИЧЕСКИЙ УНИВЕРСИТЕТ»}}\\
\hfill \break
\normalsize{Факультет электроники и вычислительной техники}\\
 \hfill \break
\normalsize{Кафедра «Системы автоматизированного проектирования и поискового конструирования»}\\
\hfill\break
\hfill \break
\hfill \break
\hfill \break
\large{Проектирование АСОиУ}\\
\hfill \break
\hfill \break
\hfill \break
\normalsize{Лабораторная работа №6}\\
\hfill \break
Перечень мероприятий, которые необходимо провести, чтобы начать предоставлять образовательные услуги   по рабочему плану ускоренного обучения\\
\hfill \break

\hfill \break
\hfill \break
\end{center}
 
\hfill \break
\hfill \break
\hfill \break
\hfill \break
\hfill \break
\hfill \break
\begin{flushright}
\normalsize{ 

Заказчик:  А.П. Пеньковская 
\\
\hfill \break
Преподаватель: А.А. Соколов 
\\
\hfill \break
Выполнили студенты группы САПР-1.н:
\\

Д.А Володина,  Д.В Корлякова  


}
\end{flushright}


\hfill \break
\hfill \break
\hfill \break
\hfill \break
\hfill \break
\hfill \break
\hfill \break
\hfill \break
\hfill \break
\hfill \break


\begin{center} Волгоград 2018 \end{center}
\thispagestyle{empty} 
 

\begin{flushleft}
\Large Перечень мероприятий, которые необходимо провести, чтобы начать предоставлять образовательные услуги   по рабочему плану ускоренного обучения (далее РПУО)
\section{Документы, которые должны поступить в подразделение кафедры, занимающееся УО до начала оформления}

\subsection{Приказ о зачислении на 1 курс в формате pdf (источник – приемная комиссия вуза);}
\subsection{Комплект документов для будущего личного дела, в который входят: }
\begin{itemize}

\item заявление о допуске к вступительным испытаниям;
\item согласие на обработку персональных данных;
\item заявление (согласие на зачисление);
\item копия диплома о высшем образовании или копия аттестата о среднем образовании (для студента);
\item копия приложения к диплому или оригинал академической справки (для студента); 
\item справка об обучении из деканата (справка о том, что абитуриет является студентом);
\item экзаменационный лист (для абитуриентов с высшим образованием);
\item письменные работы по физике, математике и русскому языку (для абитуриентов с высшим образованием);
\item договор о платных  образовательных услугах;
\item выписка из приказа о зачислении на 1 курс;
\item выписка из приказа о формировании  группы.
\end{itemize}



\section {Формирование личного дела студента} 

\subsection{Действия, которые необходимо провести до начала формирования личного дела:}
\begin{itemize}

\item закупить папки;
\item напечатать описи для наклейки на первую страницу обложки;
\item наклеить на третьей странице карман для документов;
\item провести выписку из приказов о зачислении;
\item провести выписку из приказа о формировании групп.
\end{itemize}
Необходимо использовать материалы из деканата и учитывать последовательность документов при формировании папки в порядке, указанном в пункте 1.1. После размещения всех перечисленных в п.1.1 документов документ сшивается.
Перечисленные действия необходимо провести \textit {до 1 января.}

\section{}
До \textit{1 октября} необходимо подготовить студенческие билеты (С.Б) и зачетные книжки (З.К). Для этого:
\begin{itemize}

\item из приказа о зачислении на 1 курс узнать ФИО студентов;
\item в деканате необходимо получить фотографии студентов и файл с номерами зачетных книжек; 
\item внести все полученные материалы в З.К и С.Б;
\item отнести З.К и С.Б на получение подписи и печати к декану факультета послевузовского образования;
\item отнести З.К и С.Б на получение подписи и печати к проректору по учебной работе (Гонику Игорю Леонидовичу);
\end{itemize}

\section{}
После зачисления студентов издается приказ «о формировании групп» с распределением по группам с указанием шифра группы и формы обучения (очно-заочная и заочная). Приказ издается \textit { в сентябре}, сразу после даты зачисления абитуриентов.
Название группы определяется по типу обучения и году поступления. \\*Например, название И-2.18 означается очно-заочную форму обучения 2018 года зачисления, Из-2.18 – заочную форму обучения 2018 года зачисления.


\section{}
Каждому студенту  (\textit {с сентября до ноября}) необходимо выдать:
- договор на обучение;
-дополнительное соглашение к договору (данные документы создаются Управлением маркетинга и образовательных услуг УМОУ)
- студенческий билет и зачетная книжка.


\section{Файлы, которые необходимо подготовить для подразделения УО кафедры, деканата ФПО, преподавателей и налогового отдела:}
\begin{itemize}

\item списки групп первого года обучения текущего учебного года;
\item списки групп 1 курса для учета личных дел;
\item «учет» оплаты по всем курсам (в формате xls с суммами по оплате).

\end{itemize}
Перечисленные файлы необходимо подготовить \textit {в октябре.}
\section{Файлы, которые необходимо разослать студентам:}

\begin{itemize}

\item квитанции об оплате;
\item стоимость обучения;
\item номера договоров и зачетных книжек;
\item расписание;
\item контактные данные преподавателей;
\item учебные планы «mini»;
\item файл «переаттестация».

\end{itemize}

Данные мероприятия проводятся \textit {с октября до конца учебного года.}

section {Файлы, которые необходимо получить для проведения переаттестации дисциплин учебного плана (проводится в период \textit {с 1 октября по 20 октября}): }

\begin{itemize}

\item учебные планы УО;
\item списки групп;
\item академические справки;
\item копии, приложенные к диплому;
\item шаблон для переаттестации.

\end{itemize}




\end{flushleft}
\end{document}
