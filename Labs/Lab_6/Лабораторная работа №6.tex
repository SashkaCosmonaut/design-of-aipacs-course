\documentclass[utf8x]{article}
\usepackage[utf8x]{inputenc}
\usepackage[T2A]{fontenc}
\usepackage{titlesec}
\usepackage{indentfirst} % Красная строка
\usepackage{ragged2e}

\usepackage{float}%"Плавающие" картинки
\usepackage{wrapfig}%Обтекание фигур (таблиц, картинок и прочего)
\setcounter{secnumdepth}{4}

\titleformat{\paragraph}
{\normalfont\normalsize\bfseries}{\theparagraph}{1em}{}

\usepackage[left=2cm,right=2cm,top=2cm,bottom=2cm,bindingoffset=0cm]{geometry}

\usepackage{natbib}
\usepackage{graphicx}


\begin{document}
\date{}
\begin{center}
\hfill \break
\large{Министерство образования и науки Российской Федерации}\\

\large{Государственное образовательное учреждение высшего профессионального образования}\\
\hfill \break

\large{\textbf{«ВОЛГОГРАДСКИЙ ГОСУДАРСТВЕННЫЙ ТЕХНИЧЕСКИЙ УНИВЕРСИТЕТ»}}\\
\hfill \break
\normalsize{Факультет электроники и вычислительной техники}\\
 \hfill \break
\normalsize{Кафедра «Системы автоматизированного проектирования и поискового конструирования»}\\
\hfill\break
\hfill \break
\hfill \break
\hfill \break
\large{Проектирование АСОиУ}\\
\hfill \break
\hfill \break
\hfill \break
\normalsize{Лабораторная работа №6}\\
\hfill \break
Приказы по бакалаврам по темам ВКР\\
\hfill \break

\hfill \break
\hfill \break
\end{center}
 
\hfill \break
\hfill \break
\hfill \break
\hfill \break
\hfill \break
\hfill \break
\begin{flushright}
\normalsize{ 

Заказчик:  А.П. Пеньковская 
\\
\hfill \break
Преподаватель: А.А. Соколов 
\\
\hfill \break
Выполнили студенты группы САПР-1.н:
\\

Попова С.С., Вайнгольц Н. 


}
\end{flushright}


\hfill \break
\hfill \break
\hfill \break
\hfill \break
\hfill \break
\hfill \break
\hfill \break
\hfill \break
\hfill \break
\hfill \break


\begin{center} Волгоград 2018 \end{center}
\thispagestyle{empty} 
 
\renewcommand{\contentsname}{Содержание}\tableofcontents

\newpage

\begin{flushleft}

\section{Введение}
\subsection{Наименование программы}
\justifying
Полное наименование – «Приказы по бакалаврам по темам ВКР». В дальнейшем используется краткое название – программа.
\subsection{Краткая характеристика области применения}

Деятельность руководства выпускающей кафедры, руководства университета, автоматизация документооборота.
\section{Основания для разработки}
\subsection{Документы, на основании которых ведётся разработка}

Разработка ведётся на основании заказа кафедры «Системы автоматизированного проектирования и поискового конструирования» Волгоградского государственного технического университета.
\subsection{Организация, утвердившая документ, и дата утверждения}
Документ утвердил заведующий кафедры САПР и ПК Щербаков М.В. \hfill \break
Дата утверждения документа: <<\underline{\quad}>> \underline{\quad\quad\quad\quad\quad\quad} 2018 г.

\section{Назначение разработки}

Автоматизация документооборота на кафедре САПРиПК для повышения эффективности деятельности руководства кафедры и университета при составлении приказов на темы выпускных квалификационных работ бакалавров.

\section{Требование к программе}
\subsection{Требования к функциональным характеристикам}
\subsubsection{Состав выполняемых функций}
Программа должна обеспечивать выполнение следующих функций:

\begin{itemize}

\item Формирование списка группы. В программе должна быть реализована функция формирования списка выбранной группы в электронном виде;
\item Формирование шаблона. В программе должна быть реализована функция формирования шаблона на приказ по темам выпускных квалификационных работ бакалавров в электронном виде;
\item Фрмирование служебной записки. В программе должна быть реализована функция формирования служебной записки по темам выпускных квалификационных работ бакалавров;
\item Отправка служебной записки. В программе должна быть реализована функция отправки служебной записки по темам выпускных квалификационных работ бакалавров в электронном виде ответственному за составление приказа по темам выпускных квалификационных работ бакалавров;
\item Составление приказа. В программе должна быть реализована функция формирования приказа по темам выпускных квалификационных работ бакалавров на основании ранее сформированных шаблона, списка группы и служебной записки; 
\item Печать приказа. В программе должна быть реализована функция печати ранее сформированного приказа по темам выпускных квалификационных работ бакалавров в бумажном виде.
\end{itemize}

\subsubsection{Организация входных и выходных данных}

\paragraph{Входные данные}
Входные данные: код специальности, номер группы, список группы, список тем.
\paragraph{Выходные данные}
Выходные данные: приказ по темам выпускных квалификационных работ бакалавров.

\subsection{Требования к надёжности}
\subsubsection{Требования к надёжному функционированию}

Надежное (устойчивое) функционирование программы должно быть обеспечено выполнением Заказчиком совокупности организационно-технических мероприятий, перечень которых приведен ниже:
\begin{itemize}

\item организацией бесперебойного питания технических средств;
\item испытания программных средств на наличие вредоносного программного обеспечения;
\item использованием лицензионного программного обеспечения.
\end{itemize}

\subsubsection{Время восстановления после отказа}

Время восстановления после отказа, вызванного сбоем электропитания технических средств (иными внешними факторами), не фатальным сбоем (не крахом) операционной системы, не должно превышать десяти минут при условии соблюдения условий эксплуатации технических и программных средств.

Время восстановления после отказа, вызванного неисправностью технических средств, фатальным сбоем (крахом) операционной системы, не должно превышать времени, требуемого на устранение неисправностей технических средств и переустановки программных средств.

\subsubsection{Отказы из-за некорректных действий оператора}

Отказы программы возможны вследствие некорректных действий оператора (пользователя) при взаимодействии с операционной системой. Во избежание возникновения отказов программы по указанной выше причине должна быть обеспечена работа конечного пользователя без предоставления ему административных привилегий.

\subsection{Условия эксплуатации}
Минимальное количество персонала, требуемого для работы программы, должно составлять 1 штатная единица – пользователь программы.

\subsection{Требования к составу и параметрам технических средств}

Состав технических средств, а также общесистемного и прикладного программного обеспечения:

\begin{itemize}

\item операционные системы семейства Windows не старше Windows 7;
\item процессор с минимальной тактовой частотой 1,4 ГГц, 2 ГБ оперативной памяти и 2 ГБ дискового пространства; 
\item 1С: Предприятие версии 8.2;
\item стабильное подключение к локальной сети университета;
\item монитор, мышь, клавиатура.
\end{itemize}

\subsection{Требования к информационной и программной совместимости}

\subsubsection{Требования к методам решения}

Методы решения должны обеспечивать выполнение всех этапов проектирования программы в соответствии с их порядком и сроками выполнения, указанными в разделе 7 данного документа.

\subsubsection{Требования к языкам программирования}

Исходные коды программы должны быть реализованы на языке 1С версии 8.2. В качестве интегрированной среды разработки программы должна быть использована среда 1С Конфигуратор.

\subsubsection{Требования к программным средствам, используемым программой}

Системные программные средства, используемые программой, должны быть представлены версиями операционных системы семейства Windows, начиная с Windows 7.

\section{Требования к программной документации}

В состав программной документации, сопровождающей проектируемое изделие – «Приказы по бакалаврам по темам ВКР» – необходимо включить техническое задание по ГОСТ 19.201-78.

\section{Стадии и этапы разработки}
\subsection{Стадии разработки}

Разработка должна включать следующие стадии:

\begin{itemize}

\item анализ требований пользователя (28 сентября – 31 октября);
\item разработка технического задания (2 ноября – 15 декабря); 
\item рабочее проектирование (20 декабря – 28 января);
\item реализация программы (8 февраля – 22 апреля);
\item тестирование программы (28 апреля – 15 мая).
\end{itemize}

\subsection{Этапы разработки}

На стадии анализа требований пользователя должны быть выполнены следующие этапы:

\begin{itemize}

\item изучение предметной области);
\item обзор систем-аналогов. 

\end{itemize}

На стадии разработки технического задания должны быть выполнены следующие этапы:

\begin{itemize}

\item разработка технического задания;
\item согласование и утверждение технического задания. 

\end{itemize}

На стадии рабочего проектирования должны быть выполнены перечисленные следующие этапы:

\begin{itemize}

\item разработка макетов экранных форм;
\itemразработка модели приложения; 
\item разработка алгоритмов функций, перечисленных в данном техническом задании.
\end{itemize}

На стадии реализации программы должны быть выполнены перечисленные следующие этапы:

\begin{itemize}

\item реализация вертикального прототипа;
\item доработка прототипа до конечного продукта. 

\end{itemize}

На стадии тестирования программы должны быть выполнены перечисленные следующие этапы:

\begin{itemize}

\item проверка правильности работы программы по каждой из реализованных функций;
\item анализ эффективности программы. 

\end{itemize}


\section{Порядок контроля и приёмки}
\subsection{Виды испытаний}

Испытания программы и верификация документации должны проводиться в организации заказчика. 

Приемно-сдаточные испытания программы должны производиться заведующим кафедры САПР и ПК Щербаковым М.В. 

Программа должна соответствовать всем требованиям, изложенным в техническом задании.

\subsection{Общие требования к приёмке}

Приемка программы должна производиться заведующим кафедры САПР и ПК Щербаковым М.В.

Программа должна считаться годной для приемки, если в процессе тестирования заказчиком она удовлетворяет всем пунктам данного технического задания.

\newpage
\chapter{Приложение А}

\begin{figure}[h]
\center{\includegraphics[scale=0.5]{123.jpg}}
\caption{Бизнес-процесс «Приказы по бакалаврам по темам ВКР»}
\label{fig:image}
\end{figure}


\end{flushleft}
\end{document}
