\documentclass[14pt]{extarticle}        % article, 14 pt
\usepackage[a4paper,				% one-sided article with reduced margins, a4
	    lmargin=2.5cm,rmargin=2cm,		% set left and right margins
	    tmargin=2.5cm,			% set top margin
	    bmargin=2cm,			% set bottom margin
	    marginpar=1cm,			% set margin notes width
	    marginparsep=0.5cm,			% set notes to paragraph separation width
	    headheight=17pt]{geometry}		% set head height
\usepackage{type1ec}				% use postscript cm-fonts
\usepackage[warn]{mathtext}			% use text in equations
\usepackage[T2A]{fontenc}			% internal font encoding
\usepackage[utf8]{inputenc}			% input encoding
\usepackage{caption}				% caption control
\usepackage[english,russian]{babel}		% localization
\usepackage{indentfirst}			% first indent
\usepackage{fancyhdr}				% fancy headers
\usepackage{color}				% define some colors
\usepackage{amssymb}				% AMS packages
\usepackage{amsfonts}				% AMS fonts for symbols
\usepackage{amsmath}				% AMS fonts for equations
\usepackage[integrals]{wasysym}			% symbols without integrals
\usepackage{graphicx}				% graphics including
\usepackage{array}				% basic array interface
\usepackage{tabu}				% advanced tables
\usepackage{xparse}				% document commands for rotatebox
\usepackage{titlesec}				% section titles interface
\usepackage[unicode=true,
	    colorlinks=true,
	    linkcolor=black]{hyperref}		% hyperlinks, PDF display and information options
\usepackage[utf8]{inputenc}
\usepackage{lastpage}


\setlength{\parindent}{5ex}			% set paragraph indent
\setlength{\parskip}{0pt}			% set paragraph skip
\clubpenalty=10000 \widowpenalty=10000		% single line penalty


\DeclareCaptionLabelFormat{rightparen}{#2)}	% caption format with right paren
\captionsetup{labelsep=space,justification=justified,singlelinecheck=off}
\captionsetup[table]{labelsep=period,position=t,justification=raggedright,singlelinecheck=off,aboveskip=0pt}
\captionsetup[figure]{labelsep=period,justification=centering,singlelinecheck=off,aboveskip=10pt}


\renewcommand{\labelitemi}%
	     {\normalfont\bfseries{--}}		% en-dash for itemize
\renewcommand{\le}{\leqslant}			% less-or-equal sign for russian style
\renewcommand{\ge}{\geqslant}			% graeter-then-or-equal sign for russian style
\newcommand{\degC}{\(\,^{\circ}{\rm C}\)}	% Celsius degree


% Table column formats
\newcolumntype{Y}[1]{>{\strut\hspace{0pt}}X[#1]<{\hspace{0pt}\strut}}
\tabucolumn Y
\renewcommand{\baselinestretch}{1.5}

% Rotation: \rot[<angle>][<width>]{<stuff>}
\NewDocumentCommand{\rot}{O{90} O{1em} m}{\makebox[#2][l]{\rotatebox{#1}{#3}}}%


\makeatletter
% Titlepage
\newcommand{\texzadtitle}[6]{%
\begin{titlepage}
 \begin{center}
    Министерство образования и науки Российской Федерации\\
    Федеральное государственное бюджетное образовательное учреждение высшего образования\\
    Волгоградский Государственный Технический Университет\\
    Кафедра «Системы автоматизированного проектирования и поискового конструирования»
 \end{center}
 \vspace{2cm}
 \vbox{
  \hfill
  \parbox{7.5cm}{
   \centerline{УТВЕРЖДАЮ}
   \centerline{Зав. кафедрой САПР и ПК}\vspace{20pt}
   \hrulefill~д.т.н.\,~Щербаков М.В.\par\vspace{7pt}
   \centerline{
   \hbox to 6cm{<<\rule{7mm}{0.4pt}>>\hrulefill~\number\year\,г.}}
   }
  }
 \vfill
 \begin{center}
    Разработка модуля автоматизации работы ГАК\vspace{1cm}\\
    Техническое задание\\
    \MakeUppercase{листов} \pageref{LastPage}
 \end{center}
 \vfill
 \vfill
 \vfill
 \centerline{#6}
 \vspace*{1cm}
 \centerline{Волгоград, \number\year}
\end{titlepage}
\setcounter{page}{2}
}





% Section numbering
\setcounter{secnumdepth}{5}
\renewcommand*{\thesection}{\arabic{section}}
\renewcommand*{\thesubsection}{\thesection.\arabic{subsection}}
\renewcommand*{\thesubsubsection}{\thesubsection.\arabic{subsubsection}}


\titlespacing*{\subsection}{0pt}{15pt}{1em}
\titlespacing*{\subsubsection}{0pt}{15pt}{1em}
\titlespacing*{\paragraph}{0pt}{15pt}{1em}

\titleformat{\section}[block]{\normalfont}{\hspace*{\normalparindent}\thesection}{1em}{}
\titleformat{\subsection}[block]{\normalfont}{\hspace*{\normalparindent}\thesubsection}{1em}{}
\titleformat{\subsubsection}[block]{\normalfont}{\hspace*{\normalparindent}\thesubsubsection}{1em}{}
\titleformat{\paragraph}[block]{\normalfont}{\hspace*{\normalparindent}\theparagraph}{1em}{}

\begin{document}

\texzadtitle{рабочей конструкторской документации фильтра СВЧ}%
	    {разработка РКД}%
	    {группа"~01}%
	    {группа"~12}%
	    {АФАР}%
	    {}%
	    
\newcommand{\anonsection}[1]{\section*{#1}\addcontentsline{toc}{section}{#1}}
\anonsection {Аннотация}
Техническое задание на разработку модуля автоматизации получения канцтоваров и иных товаров от поставщиков. Составлено и оформлено согласно ГОСТ 19.201-78. Страниц – \pageref{LastPage}.\par
Содержит основные сведения об объекте разработки, требования к программе и программной документации, технико-экономические показате-ли, стадии и этапы разработки, порядок контроля и приёмки.

\newpage
\tableofcontents
\newpage

\section{Введение}
\label{sec:purpose}

\subsection{Наименование программы}
Полное наименование – «Модуль автоматизации получения канцтоваров и иных товаров от поставщиков». В дальнейшем используется краткое название – программа.

\subsection{Краткая характеристика области применения}
Программа предназначена к применению в ВолгГТУ при ежегодной аттестации выпускных работ.

\newpage

\section{Основания для разработки}

\subsection{Документы, на основании которых ведется проектирование}
Разработка ведется на основании задания в рамках курса «Проектирование АСОиУ».

\subsection{Организация, утвердившая документ, и дата утверждения}
Документ утвердил зав. кафедрой САПР и ПК д.т.н. Щербаков М.В.\\
Дата утверждения документа: <<\rule{7mm}{0.4pt}>> \rule{35mm}{0.4pt} \number\year\ г.

\newpage

\section{Назначение разработки}
Автоматизированная подготовка документации, необходимой для проведения аттестации.

\newpage

\section{Требования к программе}
\subsection{Требования к функциональным характеристикам}
\subsubsection{Состав выполняемых функций}
Программа должна обеспечивать выполнение следующих функций:\par
\begin{enumerate} 
    \item Регистрация поставщика в системе\par\parindent=1cm
    При запуске системы, должна появиться форма авторизации, содер-жащая:
    \begin{itemize}
        \item текстовое поле для поиска поставщика;
        \item текстовое поле для поиска товара;
        \item кнопка регистрации поставщика;
        \item кнопка регистрации товара.
    \end{itemize}\par\parindent=1cm
    Нажатие на кнопку регистрации поставщика, должна открываться форма регистрации, содержащая чек-боксы о согласии с политикой обработки персональных данных, графическое поле для создания подписи и следующие текстовые поля:
    \begin{itemize}
        \item имя;
        \item фамилия;
        \item отчество;
        \item электронная почта;
        \item телефон;
        \item организация;
    \end{itemize}\par\parindent=1cm
    Если верные данные введены во все поля и все чек-боксы выбраны, то пользователь регистрируется в системе.
    
    \item Регистрация товара в системе.\par\parindent=1cm
    Нажатие на кнопку регистрации товара, должна открываться форма регистрации, содержащая следующие текстовые поля.
    
    \item Создание электронного документа\par\parindent=1cm
    Нажав на кнопку “создать документ”, пользователь может выбрать за-готовку документа из имеющегося в базе набора форм документов. Затем пользователь заполняет поля заготовки содержимым и сохраняет, нажав на кнопку “сохранить”.
    
    \item Отправка документа на подпись\par\parindent=1cm
    Выбрав созданный документ, пользователь, может выбрать получате-лей из базы и отправить им ссылку на документ. Получатели могут ознако-миться, с полученным документом, войдя в свои учетные записи. Также они могут выбрать поле для подписи и нажать кнопку “подписать” для вставки своей подписи.
\end{enumerate}

\subsubsection{Организация входных и выходных данных}
\paragraph{Входные данные}
Входные данные:
    \begin{enumerate}
        \item данные о поставщике;
    \end{enumerate}

\paragraph{Выходные данные}
Выходные данные:
    \begin{enumerate}
        \item данные о товаре;
    \end{enumerate}
    
\subsection{Требования к надёжности}  
\subsubsection{Требования к надёжному функционированию}
Надежное (устойчивое) функционирование программы должно быть обеспечено выполнением Заказчиком совокупности организационно-технических мероприятий, перечень которых приведен ниже:
\begin{itemize}
    \item организацией бесперебойного питания технических средств;
    \item испытания программных средств на наличие вредоносного программного обеспечения;
    \item использованием лицензионного программного обеспечения.
\end{itemize}

\subsubsection {Время восстановления после отказа}   
Время восстановления после отказа, вызванного сбоем электропита-ния технических средств (иными внешними факторами), не фатальным сбоем (не крахом) операционной системы, не должно превышать десяти минут при условии соблюдения условий эксплуатации технических и программных средств.\par
Время восстановления после отказа, вызванного неисправностью тех-нических средств, фатальным сбоем (крахом) операционной системы, не должно превышать времени, требуемого на устранение неисправностей тех-нических средств и переустановки программных средств.

\subsubsection {Отказы из-за некорректных действий оператора}
Отказы программы возможны вследствие некорректных действий оператора (пользователя) при взаимодействии с операционной системой. Во избежание возникновения отказов программы по указанной выше причине должна быть обеспечена работа конечного пользователя без предоставления ему административных привилегий.

\subsection {Условия эксплуатации}
Минимальное количество персонала, требуемого для работы про-граммы, должно составлять 1 штатная единица – пользователь программы.

\subsection {Требования к составу и параметрам технических средств}
Состав технических средств, а также общесистемного и прикладного программного обеспечения:
\begin{itemize}
    \item процессор с минимальной тактовой частотой 2.4 ГГц;
    \item объем оперативной памяти 2 ГБ; 
    \item 1 ГБ дискового пространства;
    \item монитор с разрешением 1920х1080 или больше;
    \item операционная система Windows 7 и выше;
    \item мышь и клавиатура.
\end{itemize}

\subsection {Требования к информационной и программной совместимости}
\subsubsection {Требования к методам решения}
Методы решения должны обеспечивать выполнение всех этапов проектирования программы в соответствии с их порядком и сроками выполнения, указанными в разделе 7 данного документа.
\subsubsection {Требования к языкам программирования}
Исходные коды программы должны быть реализованы на языке Python в среде разработки PyCharm Community, распространяющейся на основе бесплатной лицензии для одиночных и малых групп разработчиков Microsoft Corporation.
\subsubsection {Требования к программным средствам, используемым программой}
Для работы программного модуля необходима операционная система Microsoft Windows 7, 8, 8.1, 10 x32 или x64 с установленными библиотеками Microsoft .NET Framework 4.5.
\newpage

\section {Требования к программной документации}
В состав программной документации, сопровождающей проектируемое изделие – «Разработка программной системы для моделирования и прогнозирования наработок до отказа невосстанавливаемого промышленного оборудования»– необходимо включить техническое задание по ГОСТ 19.201-78.

\newpage

\section {Стадии и этапы разработки}
\subsection {Стадии разработки}
Разработка должна включать следующие стадии: 
\begin{itemize}
    \item анализ требований пользователя; 
    \item разработка технического задания; 
    \item рабочее проектирование;
    \item реализация программы;
    \item тестирование программы.
\end{itemize}

\subsubsection{Этапы разработки}
На стадии анализа требований пользователя должны быть выполнены следующие этапы:
\begin{itemize}
    \item изучение предметной области;
    \item обзор систем-аналогов;
\end{itemize}\par
На стадии разработки технического задания должны быть выполнены следующие этапы:
\begin{itemize}
    \item разработка технического задания;
    \item согласование и утверждение технического задания.
\end{itemize}\par
На стадии рабочего проектирования должны быть выполнены перечисленные следующие этапы:
\begin{itemize}
    \item разработка макетов экранных форм;
    \item разработка модели desktop-приложения;
    \item разработка алгоритмов функций, перечисленных в данном техническом задании.
\end{itemize}\par
На стадии реализации программы должны быть выполнены перечис-ленные следующие этапы:
\begin{itemize}
    \item реализация вертикального прототипа;
    \item доработка прототипа до конечного продукта.
\end{itemize}\par
На стадии тестирования программы должны быть выполнены пере-численные следующие этапы:
\begin{itemize}
    \item проверка правильности работы программы по каждой из реализованных функций;
    \item анализ эффективности программы.
\end{itemize}

\newpage

\section {Порядок контроля и приёмки}
\subsection {Виды испытаний}
Испытания программы и верификация документации должны прово-диться в организации заказчика.\par
Приемно-сдаточные испытания программы должны производиться зав. кафедрой САПР и ПК д.т.н. Щербаков М.В.\par
Программа должна соответствовать всем требованиям, изложенным в техническом задании.
\subsection {Общие требования к приёмке}
Приемка программы должна производиться зав. кафедрой САПР и ПК
д.т.н. Щербаков М.В.\par
Программа должна считаться годной для приемки, если в процессе те-стирования заказчиком она удовлетворяет всем пунктам данного техническо-го задания.

\label{sec:purpose}
\end{document}